\documentclass[12pt,a5paper]{report}

%%% Работа с русским языком
\usepackage{cmap}					% поиск в PDF
\usepackage{mathtext} 				% русские буквы в формулах
\usepackage{amsmath,amssymb}
\usepackage[T2A]{fontenc}			% кодировка
\usepackage[utf8]{inputenc}			% кодировка исходного текста
\usepackage[english,russian]{babel}	% локализация и переносы
\usepackage{epigraph}

\setlength\epigraphrule{0pt}
\setlength\epigraphwidth{.8\textwidth}
\renewcommand{\textflush}{flushright}
\renewcommand{\sourceflush}{flushright}

\renewcommand{\thesection}{}
\renewcommand{\thesubsection}{}

\newcommand{\textoverline}[1]{$\overline{\mbox{#1}}$}
\makeatletter
\makeatother

\author{\hl{Лёва Матюшкин}}
\date{2018}
\title{а/я}

\usepackage{tikz}
\usetikzlibrary{shapes.geometric}

\usepackage{titlesec}
\newcommand{\sectionbreak}{\clearpage}
%\usepackage[showframe]{geometry}
\usepackage[normalem]{ulem}  % для зачекивания-подчеркивания текста
% \usepackage[usenames,dvipsnames,svgnames,table,rgb]{xcolor}


\newcommand*\circled[1]{%
   \begin{tikzpicture}[baseline=(C.base)]
     \node[draw,circle,inner sep=1pt](C) {#1};
   \end{tikzpicture}}

\usepackage{titlesec}
\titleformat{\section}[display]{\normalfont \bfseries}{\hspace{0mm} \thesection}{1em}{}
\titlespacing{\section}{0pt}{3.5ex plus 1ex minus .2ex}{2.3ex plus .2ex}

\setlength\parindent{0pt}

\usepackage{hyperref}
\hypersetup{                % Гиперссылки
    unicode=true,           % русские буквы в раздела PDF
    pdftitle={а/я},         % Заголовок
    pdfauthor={Лёва Матюшкин},      % Автор
    pdfcreator={Лёва Матюшкин}, % Создатель
    colorlinks=true,        % false: ссылки в рамках; true: цветные
    linkcolor=blue,         % внутренние ссылки
    urlcolor=blue           % на URL
}
\newcommand{\hl}[1]{\colorbox{black}{#1}} % текстовыделитель

\begin{document}

\maketitle

\tableofcontents



\section[<<Когда звучит протяжная труба...>>]{* * *}
\label{kit}
Когда звучит протяжная \hyperref[mertvye]{труба}\\
в просторном коридоре перехода,\\
\\
мне холодно,\\
и хочется туда,\\
\\
где с \hyperref[kayuta]{парохода}\\
сыплется крупа\\
\\
на гладь пруда,\\
расстеленную водно,\\
\\
где рыба-кит\\
читает алфавит,\\
\\
а мы лежим\\
на дне единородно,\\
\\
жуем крупу\\
и слушаем кита.



\section{На две тишины}
\label{nadve}
На две \hyperref[kvartira]{тишины} меньше,\\
чем я хотел:\\

мальчик стучит мяч\circled{о}м,\\
а собака \colorbox{lightgray}{---} лает...\\
\newline

Ручьем\\
подпирает,\\

выталкивает\\
лучом,\\

ключом\\
скрипучую \colorbox{black}{музыку отпир}ает.



\section{Вечером}
\label{deka}
Я лежал \underline{на спине}, \textoverline{на кровати},\\
и хватал \sout{струнный} ус, словно \emph{$\widehat{yc}$} бабочки,\\
и сам был странный на вкус, \circled{г}оловой в \fbox{наволочке},\\
звук \raisebox{0.7pt}{отлипающей} от живота \raisebox{0.7pt}{\hyperref[gitarka]{деки}} принимая за шум \\ \hyperref[shkaf]{моря}.

\vspace{3cm}
\hspace{0.5cm}
\begin{tikzpicture}[domain=0:4] 
 \draw[fill=yellow!10] (7,6) -- (12,6) -- (10,4) -- (5,4) -- cycle;
\end{tikzpicture}



\section{Всадник}
Ради внутренней рифмы \raisebox{2pt}{разбил} палисадник,\\
для словесной игры --- \raisebox{-2pt}{сломал}\hspace{2pt}.\\
На невидимой \raisebox{-3pt}{лошади} \raisebox{-5pt}{плачущий} всадник\\
голубые \raisebox{-4pt}{цветы} \raisebox{-8pt}{топтал}\hspace{2pt}.\\



\section{Зеленка}
Сердце зеленкой вымазал, а все равно кровоточит. Ходишь сутулый на улицу и в буфет, спрашиваешь у него, чего же оно хочет. Сердце в ответ:\\
\\
<<Ничего не хочу, \emph{милый},\\
ничего не нужно, \emph{родной мой},\\
нет>>.

\vspace{3.5cm}

\begin{tikzpicture}[domain=0:4] 
 \draw[fill=green!10] (3,3) -- (0,3) -- (0,0) -- cycle;
  \end{tikzpicture}



\section[<<Люблю, когда все стихи...>>]{* * *}
Люблю, когда все стихи\\
лежат на одной странице,\\
\\
как темные птицы в снегу,\\
влекомые горстью руки.\\
\\
Слушать, как время шипит\\
на кончике тонкой спицы,\\
\\
которой, как осью --- полюс,\\
крутящийся мир проткну.\\
\\
Исчеркивать прямо-угольник,\\
выкраивая глубину.\\
\newpage



\section{Водефис}
\label{vodefis}
Кости, череп от скуки нарисовал на \fbox{визитке},\\
 бессмысленно долго ждал хозяина-гостя,\\
промокающего без зонта, до нитки,\\
в зеленых \hyperref[vodorosli]{\uwave{водорослях}}.

\vspace{4cm}

\begin{tikzpicture}[domain=0:4] 
 \draw[fill=green!10] (12,6) -- (7,6) -- (10,4) -- (15,4) -- cycle;
 \draw (12,6) -- (10,4);
 \draw (15,4) -- (7,6);
\end{tikzpicture}



\section{Полость}
С утомленных розовых губ \\
гость плененный читает новость: \\

совокуп- \\
ность \\

полос \\

есть \\

полость \\

(не додумывай это вглубь).


\section{Про моря, облака и реки}
\subsection{Произвол}
\emph{Вы} человек хороший...\\

Должно быть, очень милы ---\\ 
не прокляли меня сразу же,\\
не сказали \emph{пошел вон}.\\ 

Но лежим теперь с разных сторон\\ 
от полотна пилы,\\ 
и слушаем шум волн,\\
слушаем\\
\phantom{слушаем} шум\\
\phantom{слушаем шум} волн.\\ 
\\
Вы не думаете, то,\\
как облака уходят за \textoverline{горизонт},\\ 
в другую эпоху\\
назвали бы произвол?\\
\newpage



\subsection[<<Мир крошится...>>]{* * *}
\label{peski}
Мир крошится ---\\
теперь кругом пески,\\
\\
в пустыне пляжа шепот океана\\
\hl{\phantom{пропущено пропущено}} виски.\\
\\
Отняв от ряби губ\\
излом стакана,\\
\\
я вспоминаю темноту экрана,\\
когда еще ничто не началось:\\
\\
пьет воду долго лось,\\
алеет рана,\\
\\
темнеет солнце сквозь\\
разлив \hyperref[tuman]{тумана},\\
\\
и мама\\
говорит ребенку: <<Бр\'{о}сь>>.
\newpage


\subsection{Рубка}
\label{rubka}
Будь результат плохой,\\
мы не поднимем \hyperref[kisti]{трубку},\\
не постучимся в рубку,\\
не объясним пробой,\\
\\
мы побежим куда-то,\\
мы убежим куда-то,\\
воздух \hl{или} страдание\\
вспарывая головой.\\
\\
Так ходовая рубка,\\
так золотая шлюпка\\
долго стучаться будут\\
о ледяной прибой.\\
\newpage



\subsection{С холма}
\label{holma}
Муравейник,\\
строящиеся \fbox{д}\fbox{о}\fbox{м}\fbox{\circled{а}},\\
\hyperref[vodoochistki]{колючая проволока}...\\

Кинув где-нибудь велик,\\
лежа, с холма\\
разглядывать облако.

\vspace{2cm}
 \begin{tikzpicture}[domain=0:4] 
 \draw[fill=blue!10] (1,1) -- (1,2) -- (2,3) -- (3,4) -- cycle;
  \end{tikzpicture}
\newpage



\subsection{Со шкафа}
\label{shkaf}
Смотрю на \hyperref[deka]{море} со \textoverline{\fbox{шкафа}}.\\
Наклонив \circled{г}олову,\\
пристань увидел с \textoverline{полочки} для \underline{белья}.\\
\\
Пристальный, одетый с иголочки,\\
\hl{г}де \hl{б}ыл я, когда \hl{\emph{эти}} привыкли,\\
когда \hl{ст}ал я реликвией\\
вроде подреберного копья\\
\\
или сказочного апельсина,\\
закатившегося в хвойный лес...\\
\\
С платяного шкафа я слез\\
поздно, нагой и бессильный ---\\
\\
вытряхнув все из него,\\
уснул в \uwave{куче тряпья}.\\
\\
На груди из всего\\
воздуха чувствовал\\
только вес \\
тонущего корабля.
\newpage



\subsection{В каюте}
\label{kayuta}
... человек выкипел, будто чайник,\\
не доел сахарные трубочки парашютов...\\
\\
я пробрался в его \fbox{\hyperref[kit]{каюту}}\\
и вызнал \emph{тайну}.\\

\vspace{3.5cm}

\begin{tikzpicture}[domain=0:4] 
 \draw[fill=yellow!10] (4,4) -- (0,4) -- (0,0) -- cycle;
  \end{tikzpicture}
\newpage

\subsection[<<Упала в волну игла....>>]{* * *}
Упала в волну игла. \\

Может быть, море вышьет \\
то, что украл всевышний \\
из письменного стола? \\

Долго разглядывал паспорт, \\
все ли сделал прививки.\\

На обивке\\
следы нашей распри,\\
на лицах \hl{следы} улыбки.\\
\newpage

\subsection{Грозовое}

\epigraph{
  Я был смертельно рад и смертельно болен \\
тем, что, алея в поле, легла гроза, \\
что лобызая градины, шел простором, \\
и, не слезая с воздуха, осязал...
}

Грозовое облако начинается за горизонтом, \\
\\
но, будто дурная слава,\\
ближним краем идет по мне, \\

белым паром струится\\
распавшийся черный контур, \\
извиваясь на жарком солнце промеж камней. \\

$\diamond$\\

О, поскорее стемней,\\
память моя\\
о ком-то.
\newpage


\subsection{Ноябрь}
\epigraph{Век неназванного\\
и долог, и короток.\\
Безымянные облака.}

Я приехал на поезде,\\
но...\\

Был ноябрь,\\
и было поздно ---\\

облако уже улетело.\\

В прощальном жесте \hl{(пропущено)} повисла рука.\\

То ли дело --- река.\\

Звёзды \\
хо-тя бы.\\
\newpage

\subsection{Львы и тигры}
\label{bokal1}
Здесь, над рябью столов Невы,\\
с и н т е п о н о в ы е облака, \\
высей выколотых \hyperref[bokal2]{бокал} осушают, немея, львы — \\
вижу выполотые усы и подветренные бока...\\

$\diamond$\\

\hl{Ветром колыхаемые полотна}\\
\hl{изнутри закрывают конверт дождевого оконца,}\\
\hl{перевязанный лентой закатного горизонта,}\\
\hl{поцелованный сургучом уходящего солнца.}\\

Тигры лиловые надели свои медали.\\
\hl{\phantom{Пусто!}} Что будет далее?\\
Возможно ли дали этой что-то медовее ---\\
бедовей меня \hl{кого-нибудь} вы видали?
\newpage


\subsection{Зимний вечер}
\label{podvalchik}

Мне нравится, что вечер и зима,\\
что облако и неподвижный мальчик.\\

Что дальше?\\

Выйдешь из ума,\\
и вновь войдешь...\\

в какой-нибудь \hyperref[podval]{\fbox{подвальчик}}.\\
\newpage

\subsection[<<Река перевернулась в гробу...>>]{* * *}
\epigraph{
  Рыба меня похвалит --- мою блесну. \\
Время меня повалит,\\
и я усну.
}

\label{reka}

Река перевернулась в \hyperref[kamo]{гробу}\\
ею несомом,\\
\\
рот намочил губу\\
мелью и сором,\\
\\
лодка глотнула песка,\\
и \hspace{1pt}\raisebox{-2pt}{захлебнулась}\raisebox{-5pt}{ .}\\

\newpage
\epigraph{
Недостающее в~каждой вещи дополнит тоска,
в~море темном доска мерещится,
в~пироге не хватает куска.
}

\subsection[<<Вынесенному на берег рекой...>>]{* * *}

Вынесенному на берег рекой\\
медведи лицо выели,\\
\\
вы\colorbox{lightgray}{\phantom{--}}ли\\%светло-серый между вы и ли
меня спрашивали,\\
кто я такой?\\
\newline


\newpage
\subsection{Твен}
Река с шепелявым именем Миссисипи (удивительно, что в этом мире что-нибудь так называется) скована грубым льдом. И пока вы спите, мы катаемся на коньках, под огнем, в огоньках, по самую середину.\\
\\
Сегодня она вскрывается. С льдины на льдину пробираемся второпях обратно до берега...
\newpage

\subsection{Капитану корабля}

Одни бесконечности больше других,\\
и от того --- неуютнее.\\
\\
Одни вечности --- вечнее,\\
другие --- сиюминутнее.\\

$\diamond$\\

В каютной библиотечке\\
найди этот млечный стих.\\
\newpage

\subsection[<<Мы увидимся...>>]{* * *}
Мы увидимся --- отразимся по очереди в реке, \\
исказимся в пыльной, дрожа, витрине, \\

в незаметном кивке \\
гостиной,\\

друг у друга в руке.\\
\newline
\newline
$\blacktriangledown$
\newpage


\section{Про псов и собак}

\epigraph{<<Был у собаки деверь, теперь их два>> --- так говори всегда, ошибаясь дверью.}

\epigraph{Усилием воли выгрузить лес из груди. Отойди от проволоки, ну как ты пролез, отойди.}

\subsection{О домашних животных}
У меня есть собака Платон\\
и ворон Лукреций Кар.\\

От поддона струится пар ---\\
так я вижу твое <<потом>>.\\
\newpage

\subsection{Собачьи призы}
\textbf{I}\\

Нобелевскую премию мира вручили моей собаке ---\\
не знаю за что конкретно, а так ли важно ---\\
за объятия\\
жадные.\\
\newline
\newline

\textbf{II}\\

Этот пахучий приз \\
получает, гордясь, собака\\ 
за вечный вопрос в глазах \\
и облаянные облака ---\\

в гордиевых узлах \\
мусорного полубака.\\

\newpage
\subsection[Как собаки чихают]{Как собаки чихают}

\epigraph{
  
Одно от другого слова хотел бы отставить \emph{так}, --- как проходят дни, шумят деревья и ночью --- ливень. Чтобы вдруг вспомнить имя одной из восьми собак. В кашице слезных признаний откопать, чихая,\\ ореховую звезду.
}

На песью морду\\
во время чихания\\
льется аккордом\\
молчание ---\\
\\
подслушай\\
и повтори.\\
\\
Послушный\\
внутри\\
\\
и на весь мир\\
отчаянный.\\

\newpage
\subsection{Овраг}
\label{schenki}
<<Знаешь, я бы не воспринимал всё так... слишком... буквально>>.\\

На выходе из метро новый тираж собак --- \hyperref[schenok]{щенки}, кутящие вслед, развенчивают актуальность на венчики целовальные, на вензели для \hyperref[paperstone]{бумаг}.\\

Слезами целуют лоб, как проталинами --- овраг.\\ % здесь можно было бы изобразить овраг или проталины


\newpage
\subsection[<<Я в метро собака...>>]{* * *}
\label{sobaka}
\epigraph{Свежий воздух, откинутое одеяло, мятное тело и мятая простыня. 
Из глубины земли залаяла ты на меня. 
Как я теперь скучаю.}


Я в метро собака с отрезанной головой,\\
лежащая, спящая, думающая посередине вагона,\\
слушающая перестук, мерный подземный бой,\\
очень надеющаяся, что до конца перегона,\\
до остановки, платформы отрастет у нее голова,\\
голова красивая, умная, может быть, даже детская,\\
произносящая вместо лая еще не ворованные,\\
еще большие слова, заглушающие лошадиное\\
<<\colorbox{lightgray}{Пионерская}>>.\\
\\
Я в метро собака с отрезанной головой,\\
дыхание задерживаю, и держу его до конечной,\\
до бесконечной станции, в небе млечной\\
расплескавшейся тропкой, сквозь голубой\\
дозаветный свет на себя смотрю,\\
лежащую на полу полосой шерстяной\\
поперечной и что-то тебе пою,\\
мой любимый, хороший, вечный,\\
и послушный, и гордый от знания\\ 
каждой буковки людского закона,\\
но теперь, так приходится,\\
отталкивающий ногой\\
\\
собаку с откушенной мордой,\\
чтобы выбраться из вагона.\\

\newpage
\subsection{Темный лес}
Бежит собака в темный лес,\\ 
приносит брошенную палку ---\\ 
не ту, что кинута --- не жалко ---\\ 
ведь ты и сам бы не полез.\\ 
\\
Куда страшнее, что не пес\\ 
принес подгнившую корягу\\ 
и кто на оклик звонкий <<рядом>>\\ 
другое слово произнес.\\
\newline
\newline
$\blacktriangledown$





































\section{Школьные стихи}

\subsection{Попугай}
\label{popugai}

Вместимость \fbox{лифта} не превышает \hyperref[schety]{восьмерых} \emph{ч.}:\\
восьмерых человек, часов или просто --- <<буков>>...\\
\\
Попугай на \textoverline{пле}\textoverline{че}\\
\\
стал исто\colorbox{lightgray}{чн}иком слухов.
\newpage

\subsection{Аммониты}

На доске в слове \emph{вязь}\\
переговариваются аммониты,\\
и в янтарном ожерелье учительницы\\
\uwave{черный} \uline{жук}.\\
\\
Я застаю себя в отличном расположении духа.\\
\newpage

\subsection{Лас\emph{точ}ка}
Если тебя неразбуженного растереть,\\ 
\hl{должно быть,} получится отличная черная краска.\\ 
\\
Лаской мигает ласточка\\ 
(средняя треть).\\
\newline
\newline


\hl{\phantom{лас}}\colorbox{white}{\phantom{точ}}\hl{\phantom{ка}}\\
\colorbox{white}{\phantom{лас}}\hl{\phantom{точ}}\colorbox{white}{\phantom{ка}}
\newpage


\subsection[<<Пижама в серую полоску...>>]{* * *}
Пижама в \colorbox{gray}{с}е\colorbox{gray}{р}у\colorbox{gray}{ю} \colorbox{gray}{п}о\colorbox{gray}{л}о\colorbox{gray}{с}к\colorbox{gray}{у},\\

круги \circled{д\circled{е\circled{ре\circled{в}ь}е}\circled{в}} на земле,\\
\\
и --- \uline{во-оздух},\\
\phantom{и --- \uline{во-оздух},}\uwave{ме-едленный}.
\newpage

\subsection{Докладчик}
Как докладчик вздыхает. \\

Возможно, за речью его \\
скрыто больше страстей, \\
чем сегодня готов \\
исповедать.\\

Скажем, то,\\
что сильнее и больше всего\\
ждет, когда позовут отобедать.\\
\newpage

\subsection[<<Увалень дикий...>>]{* * *}
Увалень дикий \\
антенною Яги-Уда \\
\hl{случайно} меня уловил. \\

Рыдает во тьме верзила \\
у хриплой радиоточки --- \\

давно колыбельной не слышал \\
старик циклоп.\\
\newpage

\subsection[<<В состав кисти...>>]{* * *}
<<В состав кисти входят мелкие кости запястья,\\
пять длинных костей пясти\\
и кости пальцев кисти>>.\\

А когда какой-нибудь из святых\\
разламывается на части,\\
пастве лишь остается его разнести.

\vspace{1.5cm}
\begin{tikzpicture}[scale=0.5]
 \draw[fill=yellow!10] (3,9) -- (4.5,9) -- (8,4) -- (8,3) -- cycle;
 \draw[fill=red!5] (9,9) -- (6,8) -- (9,5) -- (9,5) -- cycle;
\end{tikzpicture}
\newpage



\subsection{Про букву щ}
Щека горячая и круглая щека,\\
щека щенка и щуки — щучьи щечки,\\
щека-щеколда, улица-щека,\\
и щекотание, щекотка от щипка,\\
чу! слышится пищащая щепóчка.\\

Щегол, и щи как щелок щелочной,\\
щедроты щиколок на досочках щелястых,\\
скрипящий щур, прищуренный раствор\\
темнящих глаз — щепной финал щемящий:\\

что щелканье расцепленных щелей,\\
свистящий голос, щебет ночью щебня,\\
щетина щупалец и будто бы е-ще \\
лицо щербатое украсило учебник.\\

Теперь про «У»,\\
быть может,\\
расскажу.\\


\subsection{Об отважном ученом}

В то \emph{время},\\
когда лучшим способом его измерения\\
были биения сердца,\\
\\
время каждого\\
зависело от \hl{него} самого.\\
\\
Какая должна быть смелость\\
поверить,\\
что\\
\\
и \emph{пространство}\\
вертится.\\
\newpage


\subsection{Бронзовка}
В окно, прячась дождя, залетела бронзовка,\\
сидит жирным зеленым мазком на шторе.\\

Я бы поучаствовал в разговоре,\\
но, видимо, ей неловко.\\
\newline
\newline

\colorbox{black!50!green}{\phantom{у}}
\newpage

\subsection{Ответ на письмо второклассника}

Как радостно ---\\
и письмо от души,\\
и такой крупный почерк!\\
\\
На твоих <<{\large О}>>\\
я доехал бы\\
до ЦПКиО.\\
\\
Поэтому напиши,\\
пожалуйста,\\
еще несколько стр\circled{о}чек.
\newpage

\subsection[Кувшинчик]{* * *}
Кошка переворачивает кувшинчик:\\
может быть, выплеснет что-то кроме воды.\\

Интересно дела как нынче\\
у границы черной дыры.\\
\newpage


\subsection{Жираф}
Заводной, на ключике, жираф\\
Нацепил очки --- теперь он летчик,\\ 
Вот он в самолете, длинный шарф\\
Развевает ветер. Мой дружочек,\\

\uline{Ведь} с такою шеей, головой, \\
\uline{Как} тебе хотеть еще быть выше?\\ 
\uline{Но} герой меня уже не слышит\\
\uline{И} взмывает в воздух голубой.\\
\newpage

\newpage
\subsection{Вымпел}

Вымпел ---\\
узкий длинный флаг,\\
раздвоенный на конце.\\
\\
Все вино из синяка на лице\\
выпил.\\
\newpage

\subsection{Попутчики}

Нам попались в попутчики\\
носители языка,\\
вернее --- носильщики,\\
еще точнее сказать --- грузчики:\\
\\
на скелетах кожаная листва,\\
сочные жилы, мышцы,\\
спелая голова ---\\
\\
те же\\
мученики,\\
\\
\hl{толкают} поезд,\\
\hl{и вот он} движется.\\
\\
\hl{Реже,}\\
\hl{но слышатся}\\
\hl{с разных мест}\\
диковинные слова.\\
\\
Мама закрывает уши,\\
\hl{носа пазухи и} глаза,\\
\hl{но} я, непослушный,\\
воздухом осязал,\\
\\
кожею воровал\\
что мне хотят сказать\\
\\
рваные рукава.\\
\newpage


\subsection{Правила хорошего тона}

<<Когда-нибудь>>\\
(никогда).\\
\\
Подождать\\
и кивнуть.\\
\newpage

\subsection{Транспортир}
\label{transportir}
Вместо \hyperref[sosed3]{рассветного солнца}\\
из-за линии горизонта\\
медленно вылезает\\
розовый\\
транспортир,\\
как из пенала девчачьего,\\
на котором и лев, и тапир,\\
и \hl{\phantom{полоска полоска полоска}}.
\newpage



\subsection{Ира}
\label{Ira}
Шалость была сыр\'{а} --- нам обоим хотелось с\'{ы}ра.\\
И в каком-то \fbox{окне} \fbox{\hyperref[podborodok]{двора}} постоянно кричали <<Ира>>.\\
\newpage



\subsection{Мечик}
\epigraph{
  --- Я на елке в Иркутске круглым камнем блещу, угадай мой цвет. --- Синий? --- Нет. --- Пурпур? --- Нет. --- Ну а что же? --- Ищу.
}

На предновогодней ёлке рыцарь мечик $\dagger$ свой потерял. Из всех вертикальных речек виден лишь кабель-канал $\ddagger$, да кабан на клыки поднимает ||||| штабель подарков. Мальчику стало жарко, и он \hl{---} упал. Девочка, наклонившись, шепчет: <<А мне не жа\circled{л}ко>>. <<Сэр, она не жестокая --- просто проблемы с эр>>.
\newpage

\subsection{Водоросли}
\label{vodorosli}
<<Вместо усов выросли \hyperref[vodefis]{\uwave{водоросли}}.\\
Вот тебе наказа-ание часто бриться>>.\\
\\
А лица\\
вроде бы взрослые.\\
\\
Я обратился в милицию:\\
\upbracefill\phantom{Я обратился в милицию:: }
\\
<<Фотографируют друг друга на пленку,\\
и живут потом вечно,\\
невидимые облакам>>.
\newpage


\subsection{Человек}

Мы ведь человека знаем потому только, что он спотыкается, что идет он, отталкиваясь от земли, потому только, что оставляет следы, что маленький вдалеке и очень большой вблизи, даже не умещается в голове, и сколько-нибудь людей унести можно только, когда они уменьшаются, только сказав подальше им отойти --- тогда поместятся они в объектив или книжечку, историческое ассорти, список имен и фамилий.\\
\\
Милый, чудный, родной, ты же знаешь все, расскажи.
\newpage

\subsection[<<Зевс потерял ключи...>>]{* * *}
\label{zevs}
Зевс потерял \hyperref[tuman]{ключи} от горы Олимп,\\
Гера в неудомении: <<Сколько можно!>>\\
Отмычкой сверкает молния, брошен скипетр...\\

Новые\\
\phantom{Новые}боги\\
\phantom{Новые боги}займут\\
старое\\
\phantom{старое}ложе.
\newpage


\subsection{Слез с}

Мама пришла, и помогла ему выбраться из болота,\\
автобус целый день возит мертвого пассажира.\\
Мы едем куда-то, \emph{господи}? Подскажи нам,\\
когда будем близ поворота, ---\\
\\
я использовал все \hl{свои} \hl{кар} \hl{точ} \hl{ки } проездные,\\
и теперь хочу новые.\\
\newpage

\subsection{Кисти}
\label{kisti}

Кто-нибудь \hyperref[rubka]{трубку} поднимет\\
из этих чудищ,\\
похожих на саблезубую железу:\\
\\
<<Я оловянные кисти, \hyperref[himik]{\hl{\phantom{Имя}}}, тебе везу,\\
сижу, о тебе скучая>>.\\
<<Приезжай скорее, так рад>>.\\
\\
\hl{Гош}а сделал фотоаппарат\\
из деревянной коробочки чая.\\

\vspace{1cm}
\fbox{\phantom{--}\circled{о}\phantom{--}}
\newpage

\subsection{А я}
Проснулся, а улица уже отцвела,\\
и душу мою увезли, шибанув суковатой палкой.\\
Галка вылетает \hl{медленно} из ивняка,\\
а я \circled{$\vee$} перерисовываю на кальку.\\

Видно, это страница из букваря ---\\
пока вы учили средние, я был в \fbox{больнице} ---\\
и теперь повторяю только <<А я? А я?>>,\\
пальцем небо проламывая над птицей.\\
\newpage

\subsection{Юный химик}
\label{himik}
У меня было немного денег, \\
я купил набор <<Юный химик>> \\
для идеи такой:\\

\rule{0.5pt}{50pt} хочу выгравировать на спутнике \raisebox{14pt}{твое} \raisebox{28pt}{\hyperref[kisti]{имя}.}
\newline
\newline

$\blacktriangledown$
\newpage


\section[Тебя из тебя]{* * *}

\epigraph{Я будто бы тот, кто досчитав до ста, уже никого, уже ничего не ищет --- такая приятная, знаешь ли, пустота, как если бы бог тебя из тебя вычел.}

\textbf{I}\\

Если что-то случается,\\
это еще не значит, что оно происходит:\\
\\
так числитель просачивается\\
под черту дроби\\
к знаменателю, где его возводит\\
в \hl{какую-то} степень, и --- мальчик ---\\
грудной, толстолобый,\\
в тебя возвращается,\\
через степи злобы за пальчик,\\
за руку, тебя из тебя выводит,\\
а выведя, сам заканчивается.\\
\newline
\newline

\textbf{II}\\

Когда месяц заканчивается, доходит до горла,\\
до какой-то отметки, до чего-то жестокого,\\
пытаясь предмет поднять с пола,\\
трогаешь только \emph{около},\\
\\
или вовсе упираешься кулаком,\\
упираешься дураком\\ %?
в под землею гудящий колокол.\\



\section{Три набора}
\subsection[По списку]{I}

1. Люди любят, когда по списку.\\
2. И когда встречается рифма.\\
3. А еще кораблекрушение.\\
\\
4. Никогда эти рифы так близко \\
не были в воображении.

\subsection[Набор упражнений]{II}
Этот набор упражнений\\
займет не больше пяти минут:\\

1. Прими свое поражение.\\
2. Неважно как назовут.\\
3. Ты сам по себе\\
заслуживаешь\\
уважения.\\

\subsection[Отпечаток]{III}
Оставишь ли после себя\\
мыслей своих отпечаток?\\

Нет?\\
Ну хотя бы шарф,\\
шапку, пару перчаток.\\
\newline
\newline

$\blacktriangledown$



\section{Женские стихи}
\subsection[На просьбу продавщице]{На просьбу продавщице не давать целофановый пакет, потому как мороженое съем сейчас же}

\begin{quote}
<<Я не могу иначе>>
\end{quote}
ответила она,\\
дала мне рубль сдачи,\\%семь рублей - ссылка
\\
и встала у окна,\\
смеясь тихонечко.\\
\\

\circled{ }\\

Бусина разговора оконченного\\
долго во мне сверкала.\\

\newpage\newpage
\subsection{Стихи для дворничихи}

\textbf{I}\\

Ножом срезая объявленья\\
с коробок серых и скамей,\\
ты украшаешь шорох пеньем,\\
как шею шепотом камей.\\

О дворничиха!\\
\newline
\newline

О дворничиха! Нежные стихи,\\
предположу, тебе услышать внове ---\\
не я чихал,\\
и не мои <<апчхи>>\\
благословляешь ты теперь здоровьем,\\

сметая мусор.

\newpage
\textbf{II}\\

\subsection[<<Это тяжесть мужского тела...>>]{* * *}
Это \raisebox{-2pt}{тяжесть} мужского тела, \\
груз мальчишеской теплоты. \\

Ты хотела, \\
хотела \\
ты... \\

Ну кому есть до этого дело?\\

\colorbox{lightgray}{\phantom{груз мальчишеской теплоты}} \\

\colorbox{lightgray}{\phantom{Ты хотела,}}\\
\colorbox{lightgray}{\phantom{хотела}}\\
\colorbox{lightgray}{\phantom{ты...}}\\

\newpage
\subsection{Одна пожилая женщина о щенке}
\label{schenok}

\hl{[Вот он,]} живой,\\
настоящий,\\
не боящийся\\
никого,\\

сердце щемящий,\\
светящийся,\\
мой дорогой\\
\hyperref[schenki]{щенок},\\

хвостик его\\
виляющий,\\
лижущий\\
язычок, ---\\

все, что оставил\\
мамочке\\
спившийся\\
дурачок.




\newpage
\subsection[Любимые лакомства]{У ларечка <<Любимые лакомства>>}

Пышная женщина, склонясь над орехом как белка,\\
отдает его неохотно, будто с обидой\\
на \hl{какую-то} пакость,\\
\\
\hl{словно} мы совершаем сделку,\\
ей невыгодную ---\\
\\
что-то нужно ей, кроме денег:\\
\\
сквозь баррикаду лакомств\\
выплыть в морскую пену.\\
\newpage




\subsection[<<Черный мраморный шар...>>]{* * *}
Черный мраморный шар\\ 
в белых прожилах трещин.\\
\\
Тихий медленный выдох:\\ 
\\
--- Я многое обещал,\\ 
но не больше,\\ 
\\
чем эта женщина\\ 
одним лишь своим видом.\\
\newline
\newline
$\blacktriangledown$

\newpage



\section{Ледовитые стихи}
\epigraph{К~осени время, плотнея, утончается в~мелочах, уточняется молочай и~расцвечивается полнее, и~нежнее тоска в~очах, ледяное питье ледянее.}


\textbf{I}\\

Ледовитый стакан воды\\
расщепляет свежо лучи.\\

И, оттаивая в тепле,\\
конвоир обронил ключи.\\

Цепь молчит\\
и расширен мир.\\
\newline
\newline


\textbf{II}\\

Приоткрытый рот спящего. \\
Цветы в ледовитой вазе. \\
Шепот в ночи. \\

Утром звенящие \\
привязывали \\
ключи. \\

Только молчи,\\
никому ничего \\
не рассказывай.
\newline
\newline


\textbf{III}\\

\epigraph{\emph{На панели у ворот ---\\
Талый снег,\\
На панели, у ворот ---\\
Человек.}}

\newpage
Сейчас бы вымокнуть\\
в промытом льдом саду.\\

Небрежно вымолвить:\\
ну ладно, я пойду.\\

Чтоб вслед петлял\\
колодцами наряд,\\

и пистолета\\
темный виноград\\

ловить густым\\
увертливым мешком.\\

Еще бежать,\\
идти, устав, пешком,\\

потом лежать\\
невинным двойником\\

и лед топить,\\
и так тому и быть.\\




\section{О несуществующей семье}

\epigraph{
  Помню, неслись дрожки, \\
у мамы тряслись сережки, \\
солнце переливалось\\
за край взгляда.
}

\subsection[<<У папы шнурок развязан...>>]{* * *}
У папы шнурок развязан,\\
у мамы с утра --- коса,\\
я лазал в траве,\\
и слезы\\
развязывала роса.\\
\newpage

\subsection[<<Мамина пышная юбка...>>]{* * *}
Мамина пышная юбка, \\
папиной брови щётка, \\
голубика моя голубка --- \\
нерасхоженная походка:\\

то не сладка солодка, \\
то не уксусом смочена губка.\\
\newpage\newpage

\subsection[<<Кусты цепляли за платье...>>]{* * *}

Кусты цепляли за платье,\\
как сын, узнавая птицу ---\\
заклятьем взывай такси,\\
петляющее в больницу:\\

мальчик ведь и негромкий,\\
зубы сжал о поломке.
\newpage


\subsection[<<Надломил себе руку в детстве...>>]{* * *}
Надломил себе руку в детстве ---\\
узнать из чего состою.\\
Объясняя маме последствия:\\

--- Я узнал, что сделан из боли.\\
--- Хорошо еще /не/из/любви/.
\newpage

\subsection[<<Рано утром я встану-встану...>>]{* * *}
Рано утром я встану-встану,\\
чуть рассвет, и я выйду-выйду,\\
шаг-другой за порог --- устану,\\
лягу в снег, постелю обиду,\\

прикорну, и приснится сон мне,\\
что дошел я, и стало лучше,\\
что любимую сказку вспомнил ---\\
ту, что спал, и уже не слушал.\\
\newpage

\subsection[<<Это ваш шанс плеснуть...>>]{* * *}
\epigraph{
<<Я родился и вырос...>>\\
А дальше не получилось.
}{\emph{Из резюме}}

--- Это ваш шанс плеснуть.\\
--- Вы имеете в виду блеснуть?\\
--- Плюнуть, уснуть.\\
--- Что? Я вас не понимаю.\\
--- Увянуть.\\
--- До свидания!\\
--- В добрый путь.
\newpage



\subsection{Сестричка}
Сестричка моя, ты ведь теперь веточка? Ты ведь птичка теперь, ты ведь теперь зверек? Странно и думать, что тебя может где-то не быть, глупая девочка, странно и думать, что ничто и никто тебя не сберег.
\newpage


\subsection[<<Грустью своей...>>]{* * *}
Грустью своей заслонил от себя \hl{все} живое, \\
и теперь неживое безмысленно шевелю ---\\

штору \\
велюровую,\\

матовую тишину,\\

высохший цвет левкоя.
\newpage

\subsection{Палтус}
Родители приедут с похорон,\\
привезут палтуса.\\

Тень яблони упадет в траву под окном,\\
ветер надует штор паруса...\\

--- Вы просили что-то меня подписать.\\
--- Подписать? Ах да, пожалуйста...\\
Впрочем, нет, давайте \hl{лучше} потом.
\newpage

\subsection[<<Муж и жена --- часовые...>>]{* * *}
\label{vechnoe1}
Муж и жена --- часовые карты метро:\\ % ссылка на собаку в метро
кто-то один сторожит их общие вещи,\\
кто-то чихнет --- другой говорит /\emph{будь здоров}/\\
и прочее \hyperref[vechnoe2]{вечное}.
\newpage

\subsection{Арка}
\label{arka}
\epigraph{В этом году не встретимся.\\
Увидимся годом позже.}

\epigraph{Несуществующий боже,\\
прости нам вот это все.}

\vspace{1.5cm}

Отразиться бы в арке $\bigcap$ \hyperref[paperstone]{\fbox{здания}} в полный рост. Помахать рукой удаляющейся \hyperref[avtomobili]{\fbox{машине}}. Забежать к сестрице поговорить в \hyperref[kioski]{\fbox{киоск}}. Спросить у бабушки, почему \hyperref[schety]{\fbox{счёты}} такие большие. Победить в конкурсе планетария по распознаванию зв\circled{ё}зд. Долго размышлять, для чего отец дал мне такое короткое \fbox{имя}.
\newpage


\subsection{Мармелад}
Снилось, что бабушка улицу перешла,\\ 
мне же зачем-то необходима помощь\\ 
выйти из дома\\ 
попробовать мармелад,\\ 
выторговать у продавца маргеландский овощ,\\ 
вызвонить сани для индевеющего мертвеца,\\ 
радоваться, что живой, и что скорая \hl{тоже} помощь.\\ 
\\
Тряпкой холщовой\\ 
укрыть золотого тельца.\\
\newpage


\subsection[<<Цветочек, маленький...>>]{* * *}
Цветочек, маленький, зачем тебе отцветать?\\
Некому плод унести в запутанной шерсти.\\
Видимо тот, кому было рано вставать,\\
придумал бессмертие\\
\hl{(или просто пропал без вести)}.
\newpage



\subsection{Его вариант}
Лучшее из того, что было, \\
а лучшее потому, \\
потому что ничего, \\
потому что ничего не было. \\
\\
И самое хорошее --- \\
это наш с тобой полет \\
на новую, только начатую луну \\
\\
со старой, \\
уже практически \\
не существующей,\\
ничего не значащей мебели.
\newpage

\subsection{Ее вариант}
Я, конечно бы, так не сказала, \\
я бы точно это пропела,\\

покидая \fbox{завалы} пыльные,\\
полные мебели,\\

уносясь от чада \fbox{вокзала} \\
к треску пр\circled{о}п\circled{е}лл\circled{е}ра.\\
\newpage




\subsection{Происшествие}

Это происшествие\\
скалывает рот с лица.\\
\\
Следствие рассказывает,\\
что уже двадца-\\
того,\\
двадцать-первого...\\
\\
что уже не стало отца.\\
\newpage



\subsection{О зеленой книжке} % уменьшить локально размер шрифта или начертание или поле
Я пришел к тебе, мой друг, милый одуванчик,\\
слушать хриплый граммофон да скрипеть диванчик.\\

Позвоним же всем подряд из зеленой книжки ---\\
тем, кто светел был и рад, а теперь обижен.\\

Ты, старик, как я, --- старик --- нам ходить друг к другу,\\
и в дрожащих дырках диск ковырять по кругу.\\
\newline
\newline
$\blacktriangledown$

\newpage


\section{Некоторые воспоминания о городе}


\subsection{Шателена}

По улице Шателена не ходит транспорт --\\
когда-то улица называлась \hl{П}устой переулок.

Испаряясь, рассудок, обещает, что разобрался.\\
\newpage

\subsection{Невидимые}
\epigraph{Мир, говорят, тесен --- \\
и не я один из него вырос. \\
Вспомните ли вы нас\\
в одной из\\
воскресных песен?}

Мы невидимые, мы немые,\\
пока не совершим что-то \emph{большое},\\
когда закрывающий шторы\\
потомок\\
в глубине времени\\
\\
на происходящее\\
в темном саду его предков\\
бросает последний взгляд.\\
\\
Пока длинное предложение\\
\fbox{составит} \fbox{нескромный} \fbox{ряд},\\
уже приезжает скорая.\\
\newpage



\subsection{Станция водоочистки}
\label{vodoochistki}
На неприложенном снимке вид из окна:\\

березы укрыли станцию водоочистки.\\
бетонный ворот стены, \\
ожерелье \hyperref[holma]{колючей проволоки},\\

звенящая мистика.\\
\newpage



\subsection[<<День бахчевой...>>]{* * *}
\label{kamo}

\epigraph{
Были просторные, \\
а стали внутри вещей, \\
и ты был никчемный, \\
теперь же просто ничей: \\
утверждаешь, что вещь в себе.\\
Вещь сама по себе, дуралей.}

\epigraph{
\label{kvartira}
От \hyperref[nadve]{тишины} квартиры пучеглазой я прятался в эмалевой волне, и друг \hl{неназванный} не думал обо мне и не скрывался в \hl{бесконечной} паузе.
}

День бахчевой:\\
как в тыковке живем:\\
на бечеве сушеная одежда,\\
под бечевой --- арбузный водоем...\\
\\
Бывало прежде,\\
что домой придем,\\
и сядем слушать за голодный стол\\
что накричат сегодня нам армянки\\
на скатерть-самобранку\\
через пол.\\
\\
Так сквозь стекло смотрели на простор\\
пустынной улицы,\\
на реденьких прохожих,\\
на бедненький меж ними разговор ---\\
\\
ссутулиться спешащих в коридор,\\
и слить в отхожем\\
месте кипяток,\\
не сняв пальто\\
и не разувши обувь,\\
в темнеющей прихожей\\
в форме \hyperref[reka]{гроба}.\\
\\
Календаря срывая лепесток,\\
мы говорили: <<Этот день был прежде>>,\\
читая между строк:\\
<<\emph{Камо грядеши?}>>\\

\subsection{Пена}
\label{sosed1}
\hl{Ба}тя плачет пьянющий, что раздавил котенка,\\
принесенного им самим день назад\\
в доброте душевной.\\
\\
Рот дурной покрывается желтой пеной,\\
у \hyperref[sosed2]{соседа} бубнит колонка.
\newpage


\subsection{Петербург}

\epigraph{Стану\\
бархатный,\\
мудрый,\\
будто тот\\
торт,\\
присыпанный\\
сахарной\\
пудрой
}

Лед ломовой,\\
лед пунцовый.\\

И на грязной тряпочке\\
свежий сахар.\\
\\
Постучав головой\\
леденцовой...\\
\newpage

\subsection{О памяти}

Люди заслоняют памятники на снимках,\\
становятся на их фоне,\\
пластинку\\
на патефоне\\
\\
слушают,\\
и \hl{ничего} не слышат.\\
\newpage




\subsection{Три манекена}
\label{bokal2}
Мы --- прохожие ---\\

видели облака, \\
три манекена в разбитой дождем витрине ---\\

примеряли их платья к своим голубым бокам ---\\
и \hyperref[bokal1]{бокалы} дрожали \\

нам \\
| в спины.\\
\newpage


\subsection[<<Семь рублей...>>]{* * *}
\label{schety}
\textbf{I}\\

Семь рублей за \hyperref[arka]{счет}чиком возьми. \\
Приходи в \fbox{<<коробку>>} до \hyperref[popugai]{восьми}...\\
\newline
\newline

\textbf{II}\\

Приходи однажды в никуда, \\
никогда не приходи сюда, \\
ничего сюда не приноси, \\
гой еси.\\

\newpage
\subsection{Ихтиол}
В аптеке брали ихтиол,\\
намазывали шубы,\\

и залезали на прикол\\
помиловаться в губы.\\

А дома --- с голубых икон\\
на нас чадили трубы.\\

\hl{\phantom{В аптеке брали ихтиол}},\\
\hl{\phantom{намазывали шубы}},\\

\hl{\phantom{и залезали на прикол}}\\
\hl{\phantom{помиловаться в губы}}.\\

\hl{\phantom{А дома --- с голубых икон}}\\
\hl{\phantom{на нас чадили трубы}}.\\

\newpage


\subsection[<<Глаз зашипел от снега...>>]{* * *}
Глаз зашипел от снега: белым-бело,стекло уаза проезжего щиплет калека. Мозг свело как пластинку и отлегло серебром украшений на желатине снимка. Шишки мокрые, хвоя, двойное дно, я так же кому-то понятен, как архитектура --- и правда, что некогда жили в доме одном на просп. Культуры.
\newpage

\subsection[<<Я через голову...>>]{* * *}

\begin{tabular}{ll}
Я через голову& рубашку надеваю,\\
и через голову& ботинки обуваю,\\
и через голову& в \fbox{трам}\fbox{вай} к тебе сажусь,\\
\end{tabular}
\\
но я держусь, за \textoverline{\uline{поручень}} держусь.\\
\\
В \fbox{собаке} \fbox{медленной}, \fbox{натопленной}, \fbox{бездушной},\\
коленом круглым выходящей в поворот\\ 
\uline{я} \uuline{волочусь}.
\newpage

\subsection[<<Поднимаюсь на эскалаторе...>>]{* * *}
\label{eskalator}
Поднимаюсь на эскалаторе. \\
\hl{Много вообще на чем} поднимаюсь... \\

Машут из вестибюля мне \hyperref[prosti]{ангелы}, \\
и я --- им, \\

самую малость.\\
\newpage

\newpage
\subsection{Китай-город}

И что ни день,\\
Китайгородский день:\\
\\
хрустящим яблоком\\
спокойная прогулка,\\
\\
здесь переулка\\
благодатна тень:\\
\\
густых деревьев\\
темная шкатулка,\\
\\
пустых посольств\\
немая караулка\\
\\
и тетерев,\\
токующий ячмень.
\newpage

\subsection{Туман}
\label{tuman}
Сегодня такой \hyperref[peski]{туман} ---\\
я потерял свой \fbox{дом},\\

за \hyperref[zevs]{ключами} полез в \fbox{карман},\\
и --- пропал целиком.\\


\newpage
\subsection[<<Месяцы вверяя...>>]{* * *}
\label{paperstone}
Месяцы вверяя косточкам пальцев,\\
дни отсчитывая позвонками,\\
через \fbox{\hyperref[kioski]{камень}}, \hyperref[schenki]{\fbox{бумагу}}, \fbox{ножницы}, ---\\
внутрь \raisebox{-5pt}{\hyperref[arka]{\fbox{здания.}}}

\newpage
\subsection[<<Раскрошилось печенье...>>]{* * *}
Раскрошилось печенье \\
в нагрудном кармане рубашки. \\
Кто его туда положил? \\

Тот, кто любит птиц городских, \\
а песочных печений не любит.\\


\newpage
\subsection[<<Тридцать первое...>>]{* * *} 
\epigraph{
\emph{И с тех пор все как будто больна}.}

Тридцать первое, август, среда.\\
Дождь прошел, но мне кажется темной\\
И глухой нищета этих комнат,\\
Здесь как будто случилась беда.\\

Я собаку обнял по привычке,\\
Но забыл ее глупую кличку,\\
Отпустил, и из липовой спички\\
Синий газ над плитой разжигал.\\

И стоял или что-нибудь ждал,\\
И садился в широкое кресло\\
И широкую книгу листал,\\
Но не мог ощутить ее веса,\\
\\
И поэтому снова вставал,\\
По скрипучему дереву плавал,\\
Как матрос, дикой песней забавил\\
Закипающий чайник, снимал,\\
\\
И стакан кипятку на стол ставил,\\
И на темя себе выливал.\\

\newpage
\subsection[<<Встретимся как-то...>>]{* * *}
\label{podval}
Встретимся как-то в роскошном полу\hyperref[podvalchik]{подвале},\\
чтобы сказать, как прежде, друг другу \emph{vale},\\
из Энеиды вспомнить ещё стишок...\\
Были \hyperref[temnica]{грехи} за нами --- и хорошо.

\newpage
\subsection[<<Я прохожу через камень...>>]{* * *} 
\label{kioski}
Я прохожу через \hyperref[stonephone]{камень} в густую мглу\\
сквозь неприглядную улицу в створке мозга\\
в том месте памяти,\\
где на сыром углу\\
сносят \hyperref[arka]{\fbox{к}\fbox{и}\fbox{о}\fbox{с}\fbox{к}\fbox{и}}.\\
\newline

\newpage
\subsection[<<С доски памятной стерлось...>>]{* * *}

С доски памятной стерлось --- кому она, для чего.\\
\\
Так одиночества комнаты разделяет стена тоски:\\

\begin{center}
    \fbox{одиночество бытия}\\
    \textoverline{\uline{\emph{тоска памяти}}}\\
    \fbox{одиночество небытия}\\
\end{center}
\newpage

\subsection[<<На другой стороне улицы...>>]{* * *}
Магазин на другой стороне улицы,\\
улица на другой стороне света.\\

От газеты никак не прикурится\\
оброненная сигарета.\\
\newpage

\subsection{Завещание}
Родня богатой старушки \\
шепчет: <<\colorbox{lightgray}{когда умрешь}>>. \\

\xout{Парк закрыт на просушку,}\\
\xout{и не кончается дождь.}\\
\newpage


\subsection{О длительностях целых нот}

Чайник кипел-кипел,\\
словно всю ночь нам пел,\\
нотою не меняясь,\\
\\
сам о себе крича.\\
Выкипев --- замолчал,\\
будто бы извиняясь.\\
\\
Так я вжимал звонок,\\
чтобы узнать кто умер,\\
\\
но мой тревожный зуммер\\
не разбудил никого.\\
\newpage


\newpage
\subsection[<<Прости им...>>]{* * *}
\label{prosti}
--- Прости им,\\
ибо не ведают,\\
что творят, ---\\

пустым\\
городом следуя,\\
\hyperref[eskalator]{ангелы} говорят.\\

\newpage
\subsection{Обруч}
Беги за \circled{о}бручем, мой мальчик ледяной, \\
держись за \uline{\textoverline{поручни}} в троллейбусе железном, \\
старей, и будь\\
кому-нибудь\\
полезным.

\newpage
\subsection[<<Соседи снова бранятся...>>]{* * *}
\textbf{I}\\ \label{sosed2}

\hyperref[avtomobili]{Соседи} снова бранятся.\\

Оперевшись на подоконник,\\
Собака смотрит в окно.\\
\newline
\newline

\textbf{II}\\

Ночью одна за другой \\
шелестят по шоссе машины. \\

Медленно капает кран.\\
\newline
\newline
$\blacktriangledown$



\section{Эти стихи вместе}
\subsection[<<Но не так ли устал я жить...>>]{* * *}
--- Но не так ли устал я жить,\\
что второй раз уже не буду?\\

--- Куда целовал Иуда,\\
на кукле нам покажи.\\

\subsection[<<Тем временем мертвые...>>]{* * *}
\label{mertvye}
Тем временем мертвые не спрашивают у живых\\
ни который час, ни какая теперь погода,\\
и всякий изображенный уже навсегда привык\\
ждать \hyperref[kit]{гудка} \hyperref[kayuta]{парохода}.\\



\section{Кое-что о тебе} 
\epigraph{Неподвижность певучих точек\\
на подернутой звуками глади...\\
Ты смотрела мои тетради,\\
и глаза твои гладили\\ почерк.}

\subsection[<<Светлая-светлая прядь...>>]{* * *}

Светлая-светлая прядь \\
стала зеленой.\\

Морская, стало быть, гладь \\
спрятала волны.\\

Я монетку в фонтане нашел,\\
и потратил на пристальный взгляд.

\begin{tikzpicture}[overlay] 
 \draw[fill=yellow!10] (1.5,9) -- (3,9) -- (10,3) -- (7.5,0.5) -- cycle;
\end{tikzpicture}
\newpage

\subsection[<<В этом дне ничего нет лучше...>>]{* * *}
В этом дне\\
ничего нет лучше\\
разбитых коленок, \\
зарастающих твердой коркой \\
под тонким нейлоном чулок. \\

Слушай\\
как шелестит\\
\hl{\phantom{горький}} полиэтиленовый пузырек.\\

\newpage


\subsection[<<Я отменил все встречи...>>]{* * *}
Я отменил все встречи, \\
но не будто тебя \\
любя, \\

а чтобы \\

\hl{немного} легче... \\

Как если бы \\
округлял \\

обратную сторону \\
всякой\\
вещи.\\

В этот вечер\\
я просто\\
гулял.
\newpage


\subsection[<<Не придумывай более...>>]{* * *}

Не придумывай более ничего --- \\
пусть вещь каждая сама себя означает, \\
\hl{жаждая,} вскипяти воды, выпей чаю, \\
взгляни в окно.\\
\newpage


\subsection[<<Время проходит...>>]{* * *}
\epigraph{Наперекор очевидности\\горячая струйка рта. \\
Вот и весь приговор\\твоей нерешительности.}


Время проходит, а потом сворачивает на улицу, какую не помню точно, но я держал ее под уздцы, а один из прохожих кивал, немея <<ну что за умница>>, и в руках зажатые, запотевали \fbox{дворцы}, о как горячо и влажно люди целуются, нет, конечно, можно сразу, можно не сразу, на ты.\\
\newpage




\subsection[<<Эту широкую мысль...>>]{* * *}
Эту широкую мысль \\
я хотел бы теперь воспринять:\\

написать тебе несколько писем, \\
ни одного не послать,\\

встать в очередь, и --- отвлечься, \\
проснуться заранее, и --- поспать...\\

Чистосердечный вечер,\\  %? Грубее
подсвеченная тетрадь.\\
\newpage


\subsection[<<Возьми себя с собой...>>]{* * *}
Возьми себя с собой, коль будешь \\
ты в Петербурге в должный час, — \\
возможно, встретимся. Ты любишь \\
поговорить о прежних нас, \\

погладить волосы и руки, \\
поцеловать уста в уста, \\
и снова --- через год разлуки --- \\
зайти в знакомые места. \\

На улице нас щиплет холод, \\
шипит услужливый жилец, \\
в парадной утоленный голод \\
\\
растреплет волосы... Конец, \\
ты снова едешь в дальний город,\\ 
кто слушал сказку — молодец.\\
\newpage


\subsection[<<Прямоугольником пальцев...>>]{* * *}
\epigraph{
  В глубокой зеленой траве \\
лежать холодным ручьем, \\
не думая ни о чем, \\
изредка ---
о тебе.
}

Прямоугольником пальцев\\
снял, как ты улыбаешься\\
вылетающей птице-ладони.\\

$\diamond$\\

Глядя на снимок, шепчу:\\
как наивна тогда ты была.
\newpage

\subsection[<<Чтобы кто-то звонил...>>]{* * *}
\label{vechnoe2}
Чтобы кто-то звонил узнать добрался ли ты домой, не имея на то повода поздней встречи, говорил о тебе иногда богоданное \emph{мой} или что-нибудь еще более \hyperref[vechnoe1]{вечное}.\\
\newpage


\subsection[<<Я надеюсь, тебе не будет...>>]{* * *}
\epigraph{
Не упрекай заранее за грусть\\
теперь, когда ты прочитал впервые \\
все то, что выучишь случайно наизусть \\
и повторишь в минуты роковые.}

\label{temnica}
Я надеюсь, тебе не будет\\
ничего за мои стихи ---\\
не поймают и не осудят\\
за читательские \hyperref[podval]{грехи},\\

не упрячут навек в темницу\\
из окошка спускать косу\\
за оброненную страницу\\
и упущенную слезу.\\

Может статься, и за нечтенье\\
будешь строгий держать ответ ---\\
это меньшее злоключенье:\\
нет так нет.
\newpage

\subsection[<<За подбородок...>>]{* * *}
\label{podborodok}
\epigraph{все так же жду когда не придет письмо\\
от кого не знаю об этом должно быть первое\\
но оно не пришло еще прошлой весной\\
у меня плохо с нервами}

За подбородок держу твои нервные письма. \\
Ночь тиха до безумия, \fbox{о}\fbox{к}\fbox{н}\fbox{а} во \fbox{\hyperref[Ira]{\phantom{ }двор\phantom{ }}}. \\

Скоро начатый, так же бессмысленно быстро \\
конченный разговор.\\
\newpage

\subsection{Роза и соловей}
\label{roza}
Я бы, не постеснявшись, стал бы твоим персонажем --- кем-то, с кем ты сверяешь мировозренческие часы, тем, под чьим надуманным находишься патронажем, тем, чей костер, занявшись, у ночи крадет следы. Ведь и я, не спорю, тебя себе напридумал, чтобы \hl{было} ради кого выныривать из воды...\\

$\diamond$\\

Начну писать книгу с каким-нибудь посвящением --- якобы чтобы вставить параграф о розе и соловье --- на первых страницах представлю тебя с должным почтением, через пару страниц снова упомяну о тебе, пока, наконец, просто не стану писать подряд твое имя, одно твое имя, повторяя на все лады...\\

Боже, если бы ты был вечно счастливым, \\
ты бы не был седым.
\newline
\newline
$\blacktriangledown$


\section{Про камни}

\subsection[Камни]{* * *}
Мы любили понюхать камни ---\\
из реки, из травы, из бочки,\\
из земли, от костра, из торфа.\\

Промывая, во рту сосали:\\
тот соленый, а этот горький.
\newpage

\subsection[<<Приложил к уху камень...>>]{* * *}
\label{stonephone}
Приложил к уху \hyperref[paperstone]{камень} --- дослушаю ли до конца? \\
Проверяя связь, позвонил по нему маме \\
уточнить кое-что про отца.\\
\newpage

\subsection[<<Друза кристаллов кварца...>>]{* * *}
\label{gitarka}
Друза кристаллов кварца\\
--- груда секущих жил.\\
\\
Где бы я ни служил ---\\
все время что-нибудь бряцал\\
\\
на \hyperref[deka]{гитарке},\\
на пианино.\\
\newpage

\subsection[<<И не только к другим,...>>]{* * *}
И не только к другим,\\
но к себе самому посторонний,\\
проталкивается некий через множащиеся круги,\\
потом вдруг оказывается, что то не круги, а волны,\\
а он все куда-то лез, все куда-то хотел пройти,\\

ну а в центре круга самое что ни полно,\\
там ведь, в центре круга, самый-пресамый \circled{ты}.\\
\newpage




\epigraph{У соседей гости, им пить вино, \\
говорить друг другу <<\emph{какой хороший}>>. \\
Приходи и ты — посмотреть в окно \\
на булыжную круглую площадь.
}

\subsection{На случай, если вы пропустили...}
\label{avtomobili}

Дождь несколько дней барабанил без вас по стеклу, \\
незнакомые люди захлопывались в \hyperref[arka]{\fbox{автомобили}}, \\
\hyperref[sosed3]{соседи} радовались неслучившемуся пустяку, \\

старые женщины поскальзывались у подъезда, \\
молодые девушки выпархивали в кино, \\
собака вылаивала пьяного отца семейства, \\
чтобы мама вылавливала под окном. \\

Рифмы, как видите, с каждой строкой были проще, \\
цезура тянулась дольше, спина --- нежней, \\
а потом толпа расступилась, и показалась площадь --- \\
полная площадь \hyperref[kioski]{ка}\hyperref[paperstone]{мн}\hyperref[stonephone]{ей}.\\
\newline
\newpage


\section[Бельчонок]{* * *}
\label{sosed3}
\epigraph{Не ветер осенний,
а сборщицы волокна

поют эту песню-песенку,

старина.}

Обмельчал бельчонок, сузился в труху,\\
мельник сбил бочонок, намолол муку.\\

Это мель, мой мельник. Не мука, песок.\\
Вот еженедельник, выдерни листок.\\

Приходил за солью скрюченный \hyperref[sosed1]{сосед}\\
с дымом алкоголя, чадом сигарет ---\\

узнавал бездельник, скоро ли \hyperref[transportir]{рассвет}.\\
Может, в понедельник...\\

Может быть, и нет.\\


\section[<<Собака купается в радуге поводка...>>]{На прощание}
\epigraph{
Дом стоял у ручья ---\\
у двойного смысла порога.\\
Слово было у Бога,\\
а рифма была ничья.}

Собака купается в радуге поводка,\\
в озере ошейника моет уши,\\
машет хвостом хозяевам: <<Ну, п\circled{о}ка!>>,\\
а они: <<Какая же непослушная>>.

\newpage
\null
\newpage

\center{\vfill
\href{https://www.patreon.com/matyushkin}{patreon.com/matyushkin}\\
\vspace{2.5cm}
\href{https://www.patreon.com/matyushkin}{$\nabla$}
\vfill}


\end{document}

Первая загадка: титульный лист сборника:
Не гнушаясь...
- Стихотворение А я: ссылка на первое стихотворение, болезненная неполнота речи, в первом стихотворении также упоминается алфавит
- Абонентский ящик: обложка, а за ней россыпь писем, черные прямоугольники текста - позволяют людям самим вкладывать свои письма. 
- Намеренное усложнение: дополнительные частично недоступные слои, подобие ускользающих смыслов и настроений, геометризация текста